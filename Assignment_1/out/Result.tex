\documentclass[12pt]{amsart}   	% use "amsart" instead of "article" for AMSLaTeX format
\usepackage{geometry}                		% See geometry.pdf to learn the layout options. There are lots.
\geometry{letterpaper}                   		% ... or a4paper or a5paper or ... 
%\geometry{landscape}                		% Activate for rotated page geometry
\usepackage[parfill]{parskip}    		% Activate to begin paragraphs with an empty line rather than an indent
\usepackage{graphicx}				% Use pdf, png, jpg, or eps§ with pdflatex; use eps in DVI mode
								% TeX will automatically convert eps --> pdf in pdflatex		
\usepackage{amssymb}
\usepackage{inputenc}
%SetFonts
\usepackage{amsmath}
%SetFonts

% \DeclareUnicodeCharacter{2212}{\textendash}
\usepackage{pmboxdraw}
\usepackage{newunicodechar}
\newunicodechar{└}{\textSFii}
\newunicodechar{├}{\textSFviii}
\newunicodechar{─}{\textSFx}
\begin{document}
    \title{Query results of Assignment 1}
    \author{Abhay Shankar K : cs21btech11001}
    \maketitle

    \begin{enumerate}
        \item Use table players to find the count of the number of players in each position. 

        \begin{verbatim}
        select position, count(ilkid) 
        from players 
        group by position;
        ┌──────────┬──────────────┐
        │ position │ count(ilkid) │
        ├──────────┼──────────────┤
        │ C        │ 579          │
        │ F        │ 1528         │
        │ G        │ 1465         │
        └──────────┴──────────────┘\end{verbatim}
        
        \item Find the top-5 most productive years, which is determined based on the total number 
        of games played (gp) by all the players, including both regular seasons and playoffs, for each year. 
        Solve ties by preferring chronologically older years, and print only the years.
        % 2004
        % 2003
        % 1997
        % 1995
        % 1996
        \begin{verbatim}
            
        with T as (
            select year, sum(gp) as games 
            from player_playoffs 
            group by year 
            order by games desc, year
        ),
        U as (
            select year, sum(gp) as games 
            from player_regular_season 
            group by year 
            order by games desc, year
        )
        select Y 
        from (
            select T.year as Y, T.games + U.games as G 
            from T, U 
            where T.year = U.year 
            order by G desc, Y
        )
        limit 5;
        
        ┌──────┐
        │  Y   │
        ├──────┤
        │ 2004 │
        │ 1993 │
        │ 1997 │
        │ 1995 │
        │ 1996 │
        └──────┘
        \end{verbatim}
        
        \item In the table player\_regular\_season\_career, add a new column eff (efficiency rating), which is defined as follows:
        
        	eff = (pts + reb + asts + stl + blk - ((fga - fgm) + (fta - ftm) + turnover))
        
        \begin{verbatim}
            
        alter table player_regular_season_career add eff INTEGER;
        update player_regular_season_career set eff = pts + reb + asts + 
        stl + blk + fgm + ftm - fga - fta - turnover;
        \end{verbatim}
        
        Among the players who have played more than 500 games, find the top-10 most efficient players. 
        
        \begin{verbatim}
        with T as (
            select ilkid, sum(gp) as S,sum (eff) as E
            from player_regular_season_career 
            group by ilkid
        )
        select T.ilkid 
        from T 
        where T.S > 500 
        order by T.E desc limit 10;
        
        ┌───────────┐
        │   ilkid   │
        ├───────────┤
        │ ABDULKA01 │
        │ CHAMBWI01 │
        │ MALONKA01 │
        │ MALONMO01 │
        │ GILMOAR01 │
        │ OLAJUHA01 │
        │ ROBEROS01 │
        │ ERVINJU01 │
        │ HAYESEL01 │
        │ JORDAMI01 │
        └───────────┘
        \end{verbatim} 
        % No change after foreign key shit.
        
        
        \item Find the number of players who have played more regular season games in the year 1990 than regular season games in any other year in their career. 
        
        % 57
        \begin{verbatim}
        with T as (
            select ilkid, sum(gp) as S, year 
            from player_regular_season 
            group by ilkid, year 
            order by ilkid, S
        )
        select count (*) 
        from T as _T 
        where year = 1990 and not exists (
            select * 
            from T as TT 
            where TT.ilkid = _T.ilkid and TT.year != 1990 and TT.S >= _T.S
        );
        
        ┌───────────┐
        │ count (*) │
        ├───────────┤
        │ 57        │
        └───────────┘
        \end{verbatim}
        
        \item Use table player\_regular\_season\_career to find the all-time best players. Use the two attributes gp (games played) and eff (efficiency rating) to compare players. For two players p1 and p2, we define that p1 dominates p2 if and only if p1 has a higher gp and eff value than p2.
        
        Find a set of players (ilkid, firstname, lastname, gp, eff) P, so that each player in P is not dominated by any other player in the table player\_regular\_season\_career. Return the output in ascending order of ilkid. 
        
        \begin{verbatim}
        with T as (
            select ilkid, sum(gp) as S, eff, firstname as F, lastname as L 
            from player_regular_season_career 
            group by ilkid
        )
        select * 
        from T 
        where eff in (
            select eff 
            from T 
            order by eff desc 
            limit 1
        ) or 
        S in (
            select S 
            from T 
            order by S desc 
            limit 1
        );
        
        ┌───────────┬──────┬───────┬────────┬──────────────┐
        │   ilkid   │  S   │  eff  │   F    │      L       │
        ├───────────┼──────┼───────┼────────┼──────────────┤
        │ ABDULKA01 │ 1560 │ 48247 │ Kareem │ Abdul-jabbar │
        │ PARISRO01 │ 1611 │ 30738 │ Robert │ Parish       │
        └───────────┴──────┴───────┴────────┴──────────────┘
\end{verbatim}
        
        % No change after foreign key shit.
        
    \end{enumerate}

\end{document}